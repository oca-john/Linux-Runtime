% 文档编码信息,用到的包
\documentclass[UTF8]{ctexart}		% 使用UTF8编码,文档是中文ctexart
\CTEXsetup[format={\Large\bfseries}]{section}	% 所有分级标题居左
\usepackage{fontspec}				% 用fontspec包渲染中文字体
\usepackage{geometry}				% 使用几何包
\geometry{a4paper,left=2cm,right=2cm,top=2cm,bottom=2cm}	% 定义边距


% title信息,标题、作者、日期
\title{EEG供氧面罩研究综述}
\author{Xi Zhang}
\date{2021年1月11日}


% 文档正文需要在"\begin{document}"和"\end{document}"之间
\begin{document}					% 显式声明正文开始
\maketitle							% 正文中首先要生成title,依据是上面的信息


% 摘要是正文的一部分,需要放在{document}内部
% 摘要需要在"\begin{abstract}"和"\end{abstract}"之间
\begin{abstract}
	这里是摘要占位。摘要主要介绍文章信息的总体概览。包括基于什么样的场景(背景),以及场景既有方案存在的问题(专业背景)。考虑该问题可能的解决或改进方案,制定的研究方案或策略。采用该方案或策略得到了符合或不符合预期的结果,结果是如何的。综合比较,讨论该次试验对于背景场景中问题解决得贡献,或对相关方案得优化方面的贡献。
\end{abstract}

\section{研究背景和相关研究,需在此点题}
	普遍存在于世界上的各行各业,对于高危行业从业者更是具有极大的危险性。预测有望让这一群体规避危险,但限于早期的计算能力低下,以及以往的研究策略的偏向性,一直未能实现有效的预测方法。

\section{实验设计和主要解决思路,内在逻辑需要捋清,主要方法需要交代清楚}
\subsection{实验设计}
	这里是正文第一段落占位。

\subsection{数据获取和预处理}

\subsection{特征工程和模型搭建}

\subsection{模型训练、导出、部署}

\subsection{模型测试、结果评估}

\section{结合领域背景讨论本实验对领域的贡献以及可改进之处}

\end{document}						% 显式声明正文结束
